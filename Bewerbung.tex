%%%%%%%%%%%%%%%%%%%%%%%%%%%%%%%%%%%%%%%%%%%%%%%%%%%%
%%%%%%%%%%%%%%%%%%%%%%%%%%%%%%%%%%%%%%%%%%%%%%%%%%%%
%%%%                                            %%%%
%%%%   Komplette Bewerbung in LaTeX schreiben   %%%%
%%%%                                            %%%%
%%%%%%%%%%%%%%%%%%%%%%%%%%%%%%%%%%%%%%%%%%%%%%%%%%%%
%%%%%%%%%%%%%%%%%%%%%%%%%%%%%%%%%%%%%%%%%%%%%%%%%%%%

%%%%%%%%%%%%%%%%%%%%%%%%%%%%%
%%% DOKUMENTEINSTELLUNGEN %%%
%%%%%%%%%%%%%%%%%%%%%%%%%%%%%

% Dokumentart: Zeichengröße=12pt, Blattgröße=A4, Schriftart=sans %
%%%%%%%%%%%%%%%%%%%%%%%%%%%%%%%%%%%%%%%%%%%%%%%%%%%%%%%%%%%%%%%%%%
\documentclass[12pt,a4paper,sans]{moderncv}

% Stil %
%%%%%%%%
\moderncvstyle{fancy}              % Möglichkeiten: fancy,banking,casual,oldstyle

% Farben %
%%%%%%%%%%

% Wenn man es einstellen will
    % 1. Farbe definieren wie \definiecolor{NAME}{rgb}{ZAHL,ZAHL,ZAHL} ZAHL zwischen 0-1 zb. 0.5,0.5,0.5
    % 2. Farbe laden wie \colorlet{colorX}{NAME} colorX: X kann 0,1,2 sein damit kann Farbe von Text,Zeichen oder Überschriften geändert werden

%\definecolor{SCHWARZ}{rgb}{0,0,0}% black
%\definecolor{LILA}{rgb}{0.50,0.33,0.80}% purple
%\definecolor{GRAU}{rgb}{0.45,0.45,0.45}% dark grey

%\definecolor{green}{rgb}{0, 0.9, 0}
%\colorlet{color1}{green}


%%%%%%%%%%%%%%%%%%%%
%%% PAKETE LADEN %%%
%%%%%%%%%%%%%%%%%%%%

% Deutsche Spache und Tastatur input %
%%%%%%%%%%%%%%%%%%%%%%%%%%%%%%%%%%%%%%
%\usepackage[T1]{fontenc}        % Schoene Ausgabe von Umlauten
\usepackage[utf8]{inputenc}     % Umlaute ins PDF laden
\usepackage[ngerman]{babel}     % Deutsche Sprache

% Ränder/skalieren %
%%%%%%%%%%%%%%%%%%%%
\usepackage[left=2.0cm, right=1.0cm, top=1.5cm, bottom=1.5cm]{geometry}
%\usepackage[scale=0.8]{geometry}


% Eingaben für moderncv vorbereiten %
%%%%%%%%%%%%%%%%%%%%%%%%%%%%%%%%%%%%%
\firstname{Vorname}
\familyname{Nachname}
\title{Lebenslauf}              % Für den Befehl \makecvtitle Damit der Titel der Seite TITEL ist
\address{Addresse Nummer}{PLZ ORT}{LAND \medskip}
\mobile{+99~(012)~345~6789}
\email{EMAIL@PROVIDER.KA}
%\social[linkedin]{Nutzername}
%\social[twitter]{Nutzername}
%\social[github]{Nutzername}

\nopagenumbers{}        % Keine Seitenzahlen


%%%%%%%%%%%%%%%%%%%%%%%
%%% DOKUMENT ANFANG %%%
%%%%%%%%%%%%%%%%%%%%%%%
\begin{document}


%%%%%%%%%%%%%%%%%%%%%
%%% PDF METADATEN %%%
%%%%%%%%%%%%%%%%%%%%%
\hypersetup{
    %pdftoolbar=false,
    %pdfmenubar=false,
    pdftitle={Bewerbung - VORNAME NACHNAME},                             % PDF Titel beim Aufruf im PDFreader
    pdfsubject={Bewerbung von VORNAME NACHNAME für eine STELLE},    % PDF Betreff 
    pdfkeywords={VORNAME NACHNAME} {Bewerbung} {Lebenslauf},             % Schlüsselworte für Suchen
    }


%%%%%%%%%%%%%%%%%%%
%%% ANSCHREIBEN %%%
%%%%%%%%%%%%%%%%%%%

% Angaben für den Briefkopf %
%%%%%%%%%%%%%%%%%%%%%%%%%%%%%
%% Empfänger %
%%%%%%%%%%%%%
\recipient{FIRMA}{ANSPRECHPERSON \\ ADDRESSE \\ PLZ ORT}
     %PFAD wird bei eingabe festgelegt
% Empfänger %
%%%%%%%%%%%%%
\recipient{FIRMA}{ANSPRECHPERSON \\ ADDRESSE \\ PLZ ORT}


\subject{Bewerbung um STELLE als JOB}    % Überschrift vor dem opening
\opening{Sehr geehrter Damen und Herren,}                       % Begrüßung
\closing{Mit freundlichen Grüßen, \vspace{-1cm} \\ \includegraphics[width=7.5cm]{unterschrift.png} \vspace{-3,5cm}}     % Ende mit Unterschrift
\enclosure[Anhänge]{Lebenslauf, Umschulungsthemen}          % Fußzeile um Anhänge anzugeben
\date{\today}                           % Datum

% Erstelle Briefkopf %
%%%%%%%%%%%%%%%%%%%%%%
\makelettertitle

% Inhalt des Anschreibens %
%%%%%%%%%%%%%%%%%%%%%%%%%%%
%\input{firmen/firma1/anschreiben.tex}
   %PFAD wird bei eingabe festgelegt
\input{build/firmaanschreiben.tex}

\bigskip    % Abstand zum Ende

% Ende des Anschreibens %
%%%%%%%%%%%%%%%%%%%%%%%%%
\makeletterclosing
\newpage    % Neue Seite


%%%%%%%%%%%%%%%%%%
%%% LEBENSLAUF %%%
%%%%%%%%%%%%%%%%%%

% Anfang des Lebenslaufs mit persönlichen Angaben %
%%%%%%%%%%%%%%%%%%%%%%%%%%%%%%%%%%%%%%%%%%%%%%%%%%%
\makecvtitle

% Ändere den Groß geschrieben Eintrag %
%%%%%%%%%%%%%%%%%%%%%%%%%%%%%%%%%%%%%%%

% 1. Eintrag im Lebenslauf %
%%%%%%%%%%%%%%%%%%%%%%%%%%%%

\section{Persönliche Angaben}
    \cvitem{Geburtsdatum}{GEBURTSDATUM}
    \cvitem{Geburtsort}{GEBURTSNAME}
    \cvitem{Geschlecht}{GESCHLECHT}

% 2. Eintrag im Lebenslauf %
%%%%%%%%%%%%%%%%%%%%%%%%%%%%
\section{Schulischer Werdegang}
    \cvitem{2011-2015}{KINDERGARTEN}
    \cvitem{2015-2018}{GRUNDSCHULE}
    \cvitem{2018-2020}{WEITERFÜHRENDE SCHULE}
    \cvitem{seit 02.2024}{SELBSTSTÄNDIG}

% 3. Eintrag im Lebenslauf $
%%%%%%%%%%%%%%%%%%%%%%%%%%%%
\section{Sprachen}
    \cvitem{Deutsch}{Muttersprache}
    \cvitem{Englisch}{Verhandlungssicher}
    \cvitem{Niederländisch}{Grundkenntnisse}

% 4. Eintrag im Lebenslauf %
%%%%%%%%%%%%%%%%%%%%%%%%%%%%
\section{IT-Kenntnisse}
    \cvitem{Linux}{Täglicher Einsatz gängigster Distributionen und
        Verwaltung tiefergehender Mechanismen}
    \cvitem{Netzwerke}{Fortgeschrittene Kenntnisse}
    \cvitem{Programmiersprachen}{
        \begin{itemize}
            \item Python: Fortgeschrittene Kenntnisse
            \item LaTeX: Grundlagen
        \end{itemize}
        }
    \cvitem{Virtualisierung}{Proxmox, Hyper-V, Kubernetes}

% 5. Eintrag im Lebenslauf %
%%%%%%%%%%%%%%%%%%%%%%%%%%%%
\section{Interessen}
    \cvitem{Homeserver}{In der Freizeit beschäftige ich mich mit der funktionellen
        Erweiterung des Virtuallisierungssystems um Dienste wie zb. Gamingserver,
        Codeserver zu betreiben und diese auf Sicherheitsaspekte zu testen
        }
    \cvitem{Lesen}{Besonders interessieren mich philosophische Werke von\newline Nietzsche, Camus, Sartre und Marx}
    \cvitem{Sport}{Wartung und fahren von Inlineskates und Fahrrad und Eigengewichttraining}
    \cvitem{Weiteres}{Musizieren, Malen, Reparieren von Gegenständen des täglichen Gebrauchs}


\newpage    % Neue Seite


%%%%%%%%%%%%%%%%%%%%%%%%%
%%% UMSCHULUNGSTHEMEN %%%
%%%%%%%%%%%%%%%%%%%%%%%%%

% Überschrift %
%%%%%%%%%%%%%%%
\section{\huge}\cvitem{}{\huge Umschulungsthemen}       % Text

\bigskip \bigskip   % Abstand zur Überschrift

% 1. Themenbereich %
%%%%%%%%%%%%%%%%%%%%
\section{Programmiersprachen}
    \cvitem{}{
        \begin{itemize}
            \item Strukturorientierte Programmierung
            \begin{itemize}
                \item Python
            \end{itemize}
            \item Objektorientierte Programmierung
            \begin{itemize}
                \item Python
                \item C++
            \end{itemize}
            \item UML-Diagramme
        \end{itemize}
    }

% 2. Themenbereich %
%%%%%%%%%%%%%%%%%%%%
\section{Webdevel- opment}
    \cvitem{}{
        \begin{itemize}
            \item HTML
            \item CSS
            \item PHP
        \end{itemize}
    }

% 3. Themenbereich %
%%%%%%%%%%%%%%%%%%%%
\section{Datenbanken}
    \cvitem{}{
        \begin{itemize}
            \item SQL
            \item mySQL auf Basis von MariaDB
            \item ER-Diagramme
        \end{itemize}
    }

% 4. Themenbereich %
%%%%%%%%%%%%%%%%%%%%
\section{Virtualisierung}
    \cvitem{}{
        \begin{itemize}
            \item Hyper-V
            \item KVM
        \end{itemize}
    }

% 5. Themenbereich %
%%%%%%%%%%%%%%%%%%%%
\section{Netzwerke}
    \cvitem{}{
        \begin{itemize}
            \item Planung eines Netzwerkes
            \item DHCP
            \item DNS
            \item VLan
            \item Subnetting
            \item Netzwerktopologien
            \item Strukturierte Verkabelung
        \end{itemize}
    }

% 6. Themenbereich %
%%%%%%%%%%%%%%%%%%%%
\section{Betriebs- systeme}
    \cvitem{}{
        \begin{itemize}
            \item Linux
            \begin{itemize}
                \item Debian
                \item Ubuntu
            \end{itemize}
            \item Windows 10
            \item Windows Server
            \begin{itemize}
                \item Activ Directory
                \item Domain Controller
            \end{itemize}
        \end{itemize}
    }

% 7. Themenbereich %
%%%%%%%%%%%%%%%%%%%%
\section{Weiteres}
\cvitem{}{
    \begin{itemize}
        \item Elektrotechnik
        \item Redundanz
            \begin{itemize}
                \item Raid
                \item Ausfallsicherheit
            \end{itemize}
        \item Projektmanagement
            \begin{itemize}
                \item Projektplanung und Risiken
                \item Netzpläne
                \item Gantt-Diagramme
            \end{itemize}
        \item Logik
            \begin{itemize}
                \item Logikgatter
                \item Wahrheitstabellen
            \end{itemize}
    \end{itemize}
    }


% LINUX: Hardware unter Linux, Packagemanager, Pulseaudio, Festplatten nutzung, Lizenzen, 

%%%%%%%%%%%%%%%%%%%%%%%%%%
%%% ENDE DES DOKUMENTS %%%
%%%%%%%%%%%%%%%%%%%%%%%%%%
\end{document}